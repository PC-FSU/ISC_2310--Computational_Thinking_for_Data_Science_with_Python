\documentclass{article}

\usepackage{amsmath,amsfonts,amssymb}
\usepackage[landscape]{geometry}
\usepackage{graphicx}
\usepackage{ pstricks,pst-grad  }
\usepackage{wrapfig}
\usepackage{algorithm}
\usepackage[noend]{algpseudocode}
 
\usepackage{color}
\usepackage{epstopdf}
\usepackage{framed}
 

 %  TikZ & PGF Libraries
\usepackage{tikz}
\usepackage{animate}
\usetikzlibrary{arrows}
\usetikzlibrary{positioning}
\usetikzlibrary{decorations.pathreplacing}
\usetikzlibrary{decorations.pathmorphing,patterns}
\usepackage{calc}
 

 \def \red{\color{red}}
\def \black{\color{black}}
\def \blue{\color{blue}}
\def \brown{\color{brown4}}
 \def \seag{\color{seagreen}}
\definecolor{gold}{rgb}{207,181,59}
%\newcommand{\green}{\color{thegreen}}
\definecolor{OliveGreen}{rgb}{0,0.6,0}

\def \pu{\color{purple4}}
\def \gry{\color{gray}}

\newcommand { \blueline } { \bcenter\blue\hrule width9.5in height3pt \vskip-25pt\black \ecenter }
\newcommand { \redline } { \red\hrule width9.5in height1pt\black }
\newcommand { \doubleredline } { \red\hrule width9.5in height1pt \vskip4pt \hrule width9.5in height1pt\black }
\newcommand {\redtitle}[1] {\blueline \bcenter{\bf\sffamily\red #1  }\ecenter \blueline  }
\newcommand {\bitem} {\begin{itemize} }
\newcommand {\eitem} {\end{itemize} }
\newcommand {\bdes} {\begin{description} }
\newcommand {\edes} {\end{description} }
\newcommand {\benum} {\begin{enumerate} }
\newcommand {\eenum } {\end{enumerate} }
\newcommand {\bcenter} {\begin{center} }
\newcommand {\ecenter} {\end{center} }
\newcommand {\ssp} {  \;    }
\newcommand{\vten} {\vskip10pt}
\newcommand{\vfive} {\vskip5pt}
 
 %  TIKZ
 \newcommand \btikz {\begin{tikzpicture} }
\newcommand \etikz {\end{tikzpicture} }

 


\renewcommand{\familydefault}{\sfdefault}

\textheight=6.5in
\textwidth=9.55in
\oddsidemargin = 0.0 in
\evensidemargin = 0.0 in
\topmargin = 0.0 in
\headheight = 0.0 in
\headsep = 0.0 in
\parskip = 0.2in
\parindent = 0.0in
\hfuzz=20pt
\overfullrule=0pt
%%%%%%%%%%%%%%%%%%%%%%%%%%%%%%%%%%%%%%%%%%%%%%%%


\begin{document}
\pagestyle{empty}

 \sffamily
 
\bcenter
{\large \bf Homework 5 - The Perceptron } \medskip
\ecenter

\vskip.25in
\hrule\smallskip\hrule
\medskip

%%%%%%%%%%%%%%%%%%%%%%%%
  %%%%%%%%%%%%%%%%%%%%%%%%%%%%%%%%%%%%%%%%%%%%%%%%%%%%%%
  Recall from the notes that the Perceptron model has a single artificial neuron which can have multiple inputs but with a single binary output.  Suppose that we are interested in training the algorithm to determine if a point $(x,y)$ lies  above or below  the $x$-axis.  Assume that each input  $(x,y)$ has a corresponding weight $w_x$, $w_y$ which satisfies $-1 \le w_x, w_y \le 1$.  
%%%%%
\benum
%%%%%
\item What do you think the exact values for the weights should be? (Hint: Should both weights play a role in the final solution?)
\item  For the algorithm we assume that the output is -1 if the point is below the $x$-axis and +1 if it is on or above the $x$-axis.  We set the  threshold to be 0 and calculate $t=xw_x + y w_y$ and check it against the threshold; for simplicity we ignore the bias here. We calculate the signed error as actual value - predicted value. For the learning rate use 0.01.  For starting weights take $w_x = .2$ and $w_y = -.1$.  Use the training set
$$ \{\: (1, -1), \;(1, 2 ),\; (.5,-2), \;(0,3)\; \} $$
Compute one iteration of the algorithm using the data above; output your results  in a table like the one below.   I have completed the table for the first point in the dataset so you can see what we are asking for. 
\bcenter
\begin{tabular} { || l | c | c| c| c | c |  c|| }
\hline \hline
Point &  Actual &t &  Predicted & Error  & $w_x$ & $w_y$\cr\hline &&&&&&\cr
(1,-1) & -1  &$(1)(.2) + (-1)(-.1) >0 $&  1 & -2 & .2-2( 1)(.01)=.18 & -.1-2(-1)(.01) = -.08  \cr  \hline \hline
\end{tabular}
\ecenter

 %%%%%
\eenum
%%%%%
%
 %
%%%%%%%%%%%%%%%%%%%%%%%%
 
%%%%%%%%%%%%%%%%%%%%%%%%
  \bigskip\hrule\smallskip\hrule
\bigskip

 

\end{document}

