\documentclass{article}

\usepackage{amsmath,amsfonts,amssymb}
 
\usepackage{graphicx}
\usepackage{ pstricks,pst-grad  }
\usepackage{wrapfig}
\usepackage{algorithm}
\usepackage[noend]{algpseudocode}
 
\usepackage{color}
\usepackage{epstopdf}
\usepackage{framed}
 

 %  TikZ & PGF Libraries
\usepackage{tikz}
\usepackage{animate}
\usetikzlibrary{arrows}
\usetikzlibrary{positioning}
\usetikzlibrary{decorations.pathreplacing}
\usetikzlibrary{decorations.pathmorphing,patterns}
\usepackage{calc}
 

 \def \red{\color{red}}
\def \black{\color{black}}
\def \blue{\color{blue}}
\def \brown{\color{brown4}}
 \def \seag{\color{seagreen}}
\definecolor{gold}{rgb}{207,181,59}
%\newcommand{\green}{\color{thegreen}}
\definecolor{OliveGreen}{rgb}{0,0.6,0}

\def \pu{\color{purple4}}
\def \gry{\color{gray}}

\newcommand { \blueline } { \bcenter\blue\hrule width9.5in height3pt \vskip-25pt\black \ecenter }
\newcommand { \redline } { \red\hrule width9.5in height1pt\black }
\newcommand { \doubleredline } { \red\hrule width9.5in height1pt \vskip4pt \hrule width9.5in height1pt\black }
\newcommand {\redtitle}[1] {\blueline \bcenter{\bf\sffamily\red #1  }\ecenter \blueline  }
\newcommand {\bitem} {\begin{itemize} }
\newcommand {\eitem} {\end{itemize} }
\newcommand {\bdes} {\begin{description} }
\newcommand {\edes} {\end{description} }
\newcommand {\benum} {\begin{enumerate} }
\newcommand {\eenum } {\end{enumerate} }
\newcommand {\bcenter} {\begin{center} }
\newcommand {\ecenter} {\end{center} }
\newcommand {\ssp} {  \;    }
\newcommand{\vten} {\vskip10pt}
\newcommand{\vfive} {\vskip5pt}
 
 %  TIKZ
 \newcommand \btikz {\begin{tikzpicture} }
\newcommand \etikz {\end{tikzpicture} }

 


\renewcommand{\familydefault}{\sfdefault}

\textheight=9.5in
\textwidth=6.55in
\oddsidemargin = 0.0 in
\evensidemargin = 0.0 in
\topmargin = 0.0 in
\headheight = 0.0 in
\headsep = 0.0 in
\parskip = 0.2in
\parindent = 0.0in
\hfuzz=20pt
\overfullrule=0pt
%%%%%%%%%%%%%%%%%%%%%%%%%%%%%%%%%%%%%%%%%%%%%%%%


\begin{document}
\pagestyle{empty}

 \sffamily
 
\bcenter
{\large \bf Homework 4 - K-Means Clustering  \& Logistic Regression } \medskip
\ecenter

\vskip.25in
\hrule\smallskip\hrule
\medskip

%%%%%%%%%%
\benum
%%%%%%%%%%%%%%%%%%
 \item  {\bf K-Means} \quad We want to apply the K-means algorithm  to cluster the   points below  in one dimension (i.e., on the $x$-axis) 
   $$ 2, \quad 5, \quad 10, \quad 22,\quad 26, \quad 32, \quad 42, \quad 54, \quad 66$$
   into 3 clusters where 
 the initial cluster centroids  are 0, 16 and 30.   
 
%%
 \benum
 \item Perform 3 iterations  of K-Means.  Tabulate your results in a table like the one below but give justification for each entry.    Round your results to 2 decimal places.
 Then draw the points and centroids on a line graph for each iteration indicating the clusters by circling points in each cluster.
 \medskip
 \item Has the algorithm converged using the criteria that points are no longer moving from one cluster to another?  Justify your answer.
 \eenum
 %%
 
 \bcenter
\begin{tabular}{||c ||  c| c|| c | c|| c |c||} 
\hline   \hline  
  &  \multicolumn{2}{|c|| } {Cluster 1} &\multicolumn{2}{|c|| } {Cluster 2}  &\multicolumn{2}{|c|| } {Cluster 3}   \cr \hline
 Iter. \# &  Points &Mean &  Points  & Mean  &  Points  & Mean   \cr \hline
 1     &&&&&&   \cr
  \hline
 2    &&&&&&   \cr
 \hline
3  &&&&&&   \cr \hline  
  \end{tabular}  \ecenter

 
 
 
  
\medskip \hrule \medskip
\item {\bf  Logistic Regression } \quad The general logistic function is defined by
$$ y(x) = \frac 1 {1+e^{-m(x-b)} } $$
where $x=b$ is the point where $y(b)=1/2$.   The vertical line $x=b$ is called the cutoff.  

%%%%%%%%%%%
\benum
%%%%%%%%%%%
\item  Order the functions below from the one  with the lowest cutoff value to the highest; that is, if we plotted the curves the one with the lowest cutoff would lie to the left and the highest to the right.   Justify your answers.
$$ y_1(x) = \frac 1 {1+e^{-2x -1} } \qquad    y_2(x) = \frac 1 {1+e^{-2x + 4} }  \qquad    y_3(x) = \frac 1 {1+e^{- 3(x + 4)} } 
\qquad    y_4(x) = \frac 1 {1+e^{-3x +4 } }
$$
\item Suppose we use scikit-learn's Logistic Regression model on a set of data which uses one parameter, the weight in kilograms (kg), to determine if a child between 5 and 6 years old is overweight. In this case  the denominator of the logistic function is $1 + e^{-t}$ where $t = mx + \tilde b = mx-mb$.   The algorithm returns $m$ and $\tilde b$.   Based on a training set of data the algorithm predicts $m= 1.35$, $-mb=-27.7  $ where 1 indicates True, the child is overweight and 0 indicates False, the child is not overweight.  Based on the result of the classifier,   would each of the following children be classified as  overweight or not overweight?  Explain your reasoning.
\bcenter
Child 1: 19.9 kg \qquad Child 2:  20.7kg \qquad Child 3: 23.9 kg \qquad Child 4: 20.1 kg
\ecenter
\item Suppose we use Logistic Regression on a set of data which uses two parameters, the weight in kilograms (kg) and the height in centimeters (cm), to determine if a child between 5 and 6 years old is overweight.  Because there are two features in this dataset the  denominator of the logistic function is $1 + e^{-t}$ where $t =m1*x + m2*y + \tilde b$. where $x,y$ are the two features.  The algorithm returns  $m_1, m_2, \tilde b$.  

Assume that the algorithm  returns the values {\red $m_1 = 0.082, m_2 = 1.375$ and $\tilde b = -37.13$}
%%%%%
\benum
%%%%%
\item  Write the equation of the line where the height $h$ is on the $x$-axis and weight on the $y$-axis; that is write in the form of $w(h) = mh + b$ where $m$ is the slope, $h$ is the height and $b$ is the intercept.  
\item
  Based on the result of the classifier,   would each of the following children be classified as  overweight or not overweight?
\bcenter
Child 1:  98 cm, 19.9 kg   \qquad Child 2:  112 cm, 20.7 kg    
\ecenter
%%%%%
\eenum
%%%%%

%%%%%%%%%%%
\eenum
%%%%%%%%%%%
%
%
\eenum

%%%%%%%%%%%%%%%%%%%%%%%%%%%%%
{  
\end{document}

